\documentclass[11pt]{article}

\usepackage{amsmath}
\usepackage{textcomp}

% Add other packages here %


% Put your group number and names in the author field %
\title{\bf Excercise 3\\ Implementing a deliberative Agent}
\author{Group \textnumero : Student 1, Student 2}


% N.B.: The report should not be longer than 3 pages %


\begin{document}
\maketitle

\section{Model Description}

\subsection{Intermediate States}
% Describe the state representation %

\subsection{Goal State}
% Describe the goal state %

\subsection{Actions}
% Describe the possible actions/transitions in your model %


\section{Implementation}

\subsection{BFS}
% Details of the BFS implementation %

\subsection{A*}
% Details of the A* implementation %

\subsection{Heuristic Function}
% Details of the heuristic functions: main idea, optimality, admissibility %


\section{Results}

\subsection{Experiment 1: BFS and A* Comparison}
% Compare the two algorithms in terms of: optimality, efficiency, limitations %
% Report the number of tasks for which you can build a plan in less than one minute %

\subsubsection{Setting}
% Describe the settings of your experiment: topology, task configuration, etc. %

\subsubsection{Observations}
% Describe the experimental results and the conclusions you inferred from these results %


\subsection{Experiment 2: Multi-agent Experiments}
% Observations in multi-agent experiments %

\subsubsection{Setting}
% Describe the settings of your experiment: topology, task configuration, etc. %

\subsubsection{Observations}
% Describe the experimental results and the conclusions you inferred from these results %

\end{document}